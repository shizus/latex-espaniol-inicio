\documentclass[spanish] {article}
\usepackage[T1]{fontenc}
\usepackage{selinput}
\usepackage{hyperref}
\SelectInputMappings{%
  aacute={á},
  eacute={é},
  iacute={í},
  oacute={ó},
  uacute={ú},
  ntilde={ñ},
  Euro={€}
}
\usepackage{babel}
\begin{document}
\title{Prueba de Oposición de Ayudantes 2da}
\author{Gabriel La Torre}

\maketitle
\newpage
\section{Introducción}
A continuación se dará un enunciado de un ejericio de la guía de Ondas Mecánicas. Posteriormente se explicarán los supuestos a considerar antes de resolverlo, como por ejemplo, el conocimiento previo de los alumnos al momento de presentarles el ejercicio en cuestión.
Finalmente se resolverá el ejercicio con el agregado de algunas notas explicativas que den al lector una idea de la explicación que se dará en la clase.
\section{Enunciado}
La expresión de una cierta onda es $y = 10 sen [2 \pi (2x- 100 t)]$, donde x está en metros y t en 
segundos. Halle: 
\begin{enumerate}
\item La amplitud
\item La longitud de onda
\item La frecuencia
\item La velocidad de propagación de la onda. 
\item Trace un diagrama de la onda en el que se muestre la amplitud y la longitud de onda. 
\end{enumerate}

\section{Supuestos}
Al momento de presentarles este ejercicio, los alumnos ya tuvieron la introducción teórica a las ondas mecánicas. Saben identificar una función de onda, han sido informados de la definición de una onda y la diferencia entre ondas mecánicas y ondas electromagnéticas.
\section{Resolución}
Primero analizamos la fórmula
$$y = 10 sen [2 \pi (2x- 100 t)]$$
La forma de la función cumple con la forma de una función de onda
$$y = A sen (kx- \omega t + \varphi)]$$
De solo ver la forma podemos saber que la amplitud es 10.
$$ \frac{\pi}{4} = \int_0^1 \frac{1}{1+x^2} dx $$


http://mate.dm.uba.ar/~pdenapo/tutorial-latex/node2.html
\newpage
\section{Referencias}

\begin{itemize}
  \item \href{http://mate.dm.uba.ar/~pdenapo/tutorial-latex/node2.html}{Pablo Luis De Nápoli}.
\end{itemize}



\end{document}
