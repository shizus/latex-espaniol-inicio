\documentclass[spanish] {article}
\usepackage[T1]{fontenc}
\usepackage{selinput}
\usepackage{hyperref}
\SelectInputMappings{%
  aacute={á},
  eacute={é},
  iacute={í},
  oacute={ó},
  uacute={ú},
  ntilde={ñ},
  Euro={€}
}
\usepackage{babel}
\begin{document}
\title{Prueba de Oposición de Ayudantes 2da}
\author{Gabriel La Torre}

\maketitle
\newpage
\section{Introducción}
La expresión de una cierta onda es $y = 10 sen [2 \pi (2x- 100 t)]$, donde x está en metros y t en 
segundos. Halle: la amplitud, la longitud de onda, la frecuencia y la velocidad de propagación de la 
onda. Trace un diagrama de la onda en el que se muestre la amplitud y la longitud de onda. 
\section{Una fórmula}
Una fórmula con \TeX es mejor:
$$ \frac{\pi}{4} = \int_0^1 \frac{1}{1+x^2} dx $$


http://mate.dm.uba.ar/~pdenapo/tutorial-latex/node2.html
\newpage
\section{Referencias}

\begin{itemize}
  \item \href{http://mate.dm.uba.ar/~pdenapo/tutorial-latex/node2.html}{Pablo Luis De Nápoli}.
\end{itemize}



\end{document}
